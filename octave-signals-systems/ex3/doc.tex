\documentclass{article}
\usepackage[utf8]{inputenc}
\usepackage[greek,english]{babel} \usepackage{alphabeta}
\usepackage{fancyhdr}
\usepackage{listings}
\usepackage{mathtools}
\usepackage{xcolor}
\usepackage{graphicx}
\usepackage{float}
\usepackage[backend=biber]{biblatex}

\title{Σήματα και Συστήματα - Εργασία 3}
\author{Χρήστος Μαργιώλης - 19390133}
\date{Απρίλιος 2021}

\begin{document}

\begin{titlepage}
	\maketitle
	\begin{figure}[t!]
	\begin{center}
	\includegraphics[scale=0.3]{./res/uniwalogo.png} \\
	\Large
	\textbf{Πανεπιστήμιο Δυτικής Αττικής} \\
	\large
	Τμήμα Μηχανικών Πληροφορικής και Ηλεκτρονικών Υπολογιστών
	\end{center}
	\end{figure}
\end{titlepage}

\renewcommand{\contentsname}{Περιεχόμενα}
\tableofcontents

\section{'Ασκηση 1}

\begin{itemize}
	\item 'Εστω το σύστημα ολοκληρωτή που φαίνεται στο παρκάτω
		σχήμα (φυλλάδιο άσκηση σελίδα 12). 'Ενα τέτοιο
		σύστημα δέχεται ως είσοδο ένα σήμα $x(t)$ και η έξοδος
		$y(t)$ δίνεται από την σχέση:
		\[y(t) = \int_{-\infty}^{t} x(τ)dτ\]
		Σας ζητείται να μελετήσετε αυτό το σύστημα και να
		απαντήσετε ως προς τις ιδιότητες:
		\begin{itemize}
			\item Γραμμικό ή μη γραμμικό.
			\item Δυναμικό ή στατικό.
			\item Αιτιατό ή μη αιτιατό.
			\item Χρονικά αμετάβλητο ή χρονικά μεταβαλλόμενο.
			\item Ευσταθές ή ασταθές.
		\end{itemize}
\end{itemize}

<++>

\section{'Ασκηση 2}

\begin{itemize}
	\item Να εξετάσετε ως προς την ευστάθεια το σύστημα με σχέση
		εισόδου-εξόδου $y(t) = e^{x(t)}$.
\end{itemize}

<++>

\section{'Ασκηση 3}

\begin{itemize}
	\item Να εξεταστεί εάν το σύστημα που διέπεται από την σχέση
		εισόδου-εξόδου $y(t) = x(t/4)$ είναι αιτιατό.
\end{itemize}

<++>

\end{document}
