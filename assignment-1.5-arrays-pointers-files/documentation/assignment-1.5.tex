\documentclass{article}
\usepackage[utf8]{inputenc}
\usepackage[greek,english]{babel}
\usepackage{alphabeta}
\usepackage{fancyhdr}
\usepackage{listings}
\usepackage{xcolor}
 
% Settings

\definecolor{codered}{rgb}{0,0.6,0}
\definecolor{codegray}{rgb}{0.5,0.5,0.5}
\definecolor{codegreen}{rgb}{0.0,0.66,0.42}
\definecolor{hanblue}{rgb}{0.27,0.42,0.81}
\definecolor{codeorange}{rgb}{1.0,0.55,0.0}

\lstdefinestyle{mystyle}
{
    commentstyle=\color{codegray},
    otherkeywords=
    {
        >,<,.,;,-,!,=,~,&,*,+,-,
        int8_t, % Add more types
        printf, scanf, exit % Add more functions
    },
    keywordstyle=\color{hanblue},
    numberstyle=\tiny\color{codegray},
    stringstyle=\color{codegreen},
    basicstyle=\ttfamily\small,
    breakatwhitespace=false,         
    breaklines=true,                 
    captionpos=b,                    
    keepspaces=true,                 
    numbers=left,                    
    numbersep=5pt,                  
    showspaces=false,                
    showstringspaces=false,
    showtabs=false,                  
    tabsize=4
}

% Colors the digits
\lstset{literate=
   *{0}{{{\color{red!20!violet}0}}}1
    {1}{{{\color{red!20!violet}1}}}1
    {2}{{{\color{red!20!violet}2}}}1
    {3}{{{\color{red!20!violet}3}}}1
    {4}{{{\color{red!20!violet}4}}}1
    {5}{{{\color{red!20!violet}5}}}1
    {6}{{{\color{red!20!violet}6}}}1
    {7}{{{\color{red!20!violet}7}}}1
    {8}{{{\color{red!20!violet}8}}}1
    {9}{{{\color{red!20!violet}9}}}1
}
\lstset{style=mystyle}

\pagestyle{fancy}
\fancyhf{}

% Document starts here

\rhead{Εργασία 5}
\lhead{Πίνακες - Δείκτες - Αρχεία}
\rfoot{\thepage}

\title{Εργασία 5: Πίνακες - Δείκτες - Αρχεία}
\author{Χρήστος Μαργιώλης - Εργαστηριακό τμήμα 9}
\date{Ιανουάριος 2020}

\begin{document}

\begin{titlepage}
    \maketitle
\end{titlepage}

\renewcommand{\contentsname}{Περιεχόμενα}
\tableofcontents

\section{Δομή προγραμμάτων και οδηγίες εκτέλεσης}

    \subsection{Εκτέλεση από Linux}
\begin{lstlisting}[language=bash]
$ cd path-to-program
$ make
$ make run
$ make run ARGS=txt/data.txt #fcombinations ONLY
$ make clean #optional
\end{lstlisting}

    \subsection{Δομή φακέλων}

    Το κάθε πρόγραμμα, είναι δομημένο ως εξής: Υπάρχουν πέντε φάκελοι, καθώς και ένα Makefile
    στο top directory. Στον φάκελο src βρίσκονται οι πηγαίοι κώδικες, στον include τα header
    files, στον οbj τα object files και στον bin το εκτελέσιμο αρχείο. Στον φάκελο txt
    υπάρχουν τα text files που διαβάζονται οι γράφονται από το κάθε πρόγραμμα.
    Το Makefile είναι υπεύθυνο για την μεταγλώττιση όλων των αρχείων μαζί, και την τοποθέτησή
    τους στους κατάλληλους φακέλους, την εκτέλεση των προγραμμάτων, καθώς και τον καθαρισμό των
    φακέλων (διαγράφει τα object files και το εκτελέσιμο με την εντολή make clean).

\section{combinations - συνδυασμοί}

    \subsection{main.c}
        \lstinputlisting[language=C]{../combinations/src/main.c}

    \subsection{combinations.c}
        \lstinputlisting[language=C]{../combinations/src/combinations.c}

    \subsection{combinations.h}
        \lstinputlisting[language=C]{../combinations/include/combinations.h}

    \subsection{arrhandler.c}
        \lstinputlisting[language=C]{../combinations/src/arrhandler.c}

    \subsection{arrhandler.h}
        \lstinputlisting[language=C]{../combinations/include/arrhandler.h}

    \subsection{Διάγραμμα ροής}

    \subsection{Περιγραφή υλοποιήσης}


\section{kcombinations - συνδυασμοί με K}

    \subsection{main.c}
        \lstinputlisting[language=C]{../kcombinations/src/main.c}

    \subsection{kcombinations.c}
        \lstinputlisting[language=C]{../kcombinations/src/kcombinations.c}

    \subsection{kcombinations.h}
        \lstinputlisting[language=C]{../kcombinations/include/kcombinations.h}

    \subsection{arrhandler.c}
        \lstinputlisting[language=C]{../kcombinations/src/arrhandler.c}

    \subsection{arrhandler.h}
        \lstinputlisting[language=C]{../kcombinations/include/arrhandler.h} 

    \subsection{Διάγραμμα ροής}

    \subsection{Περιγραφή υλοποιήσης}


\section{fcombinations - συνδυασμοί από αρχείο}

    \subsection{main.c}
        \lstinputlisting[language=C]{../fcombinations/src/main.c}

    \subsection{fcombinations.c}
        \lstinputlisting[language=C]{../fcombinations/src/fcombinations.c}

    \subsection{fcombinations.h}
        \lstinputlisting[language=C]{../fcombinations/include/fcombinations.h}

    \subsection{arrhandler.c}
        \lstinputlisting[language=C]{../fcombinations/src/arrhandler.c}

    \subsection{arrhandler.h}
        \lstinputlisting[language=C]{../fcombinations/include/arrhandler.h} 

    \subsection{Διάγραμμα ροής}

    \subsection{Περιγραφή υλοποιήσης}


\section{minesweeper - ναρκαλιευτής}

    \subsection{main.c}
        \lstinputlisting[language=C]{../ncurses-minesweeper/src/main.c}

    \subsection{minesweeper.c}
        \lstinputlisting[language=C]{../ncurses-minesweeper/src/minesweeper.c}

    \subsection{minesweeper.h}
        \lstinputlisting[language=C]{../ncurses-minesweeper/include/minesweeper.h}

    \subsection{gameplay.c}
        \lstinputlisting[language=C]{../ncurses-minesweeper/src/minesweeper.c}

    \subsection{gameplay.h}
        \lstinputlisting[language=C]{../ncurses-minesweeper/include/gameplay.h}

    \subsection{navigation.c}
        \lstinputlisting[language=C]{../ncurses-minesweeper/src/navigation.c}

    \subsection{navigation.h}
        \lstinputlisting[language=C]{../ncurses-minesweeper/include/navigation.h}

    \subsection{settings.c}
        \lstinputlisting[language=C]{../ncurses-minesweeper/src/settings.c}

    \subsection{settings.h}
        \lstinputlisting[language=C]{../ncurses-minesweeper/include/settings.h}

    \subsection{outputs.c}
        \lstinputlisting[language=C]{../ncurses-minesweeper/src/outputs.c}

    \subsection{outputs.h}
        \lstinputlisting[language=C]{../ncurses-minesweeper/include/outputs.h}

    \subsection{wins.c}
        \lstinputlisting[language=C]{../ncurses-minesweeper/src/wins.c}

    \subsection{wins.h}
        \lstinputlisting[language=C]{../ncurses-minesweeper/include/wins.h}

    \subsection{Διάγραμμα ροής}

    \subsection{Περιγραφή υλοποιήσης}

    Ο ναρκαλιευτής αυτός χρησιμοποιεί την βιβλιοθήκη ncurses και είναι δομημένος ως εξής:
    Από το main.c καλούνται αρχικά οι συναρτήσεις δημιουργίας των παραθύρων που θα εμφανιστούν
    στην οθόνη και στην συνέχεια καλούνται οι συναρτήσεις δημιουργίας των πινάκων $Μ \times N$,
    για το ναρκοπέδιο και για τον πίνακα που έχει "κρυμμένα" τα κελιά αντίστοιχα.
    Τέλος από την main καλείται η συνάρτηση που θα ξεκινήσει το παιχνίδι.  

    Οι συναρτήσεις για τις στήλες, γραμμές, και αριθμό των ναρκών βρίσκονται στο settings.h  
    % όρια διαστάσεων κλπ

    Στο minesweeper.c εκτελούνται όλες οι συναρτήσεις δημιουργίας πινάκων, τοποθέτησεις ναρκών,
    μέτρημα των βομβών στα γειτονικά κελιά, καθώς και γέμισμα των κενών θέσεων τους.  

    Έπειτα, στο gameplay.c εκτελείται το παιχνίδι - αρχικά τυπώνεται ο πίνακας και το περίγραμμα που
    υπάρχει ανάμεσα σε κάθε κελί ώστε να είναι πιο εμφανίσιμο και πιο εύχρηστο το παιχνίδι. Προκειμένου
    τα κελιά να τοποθετηθούν στις κατάλληλες θέσεις στον πίνακα, δηλαδή να είναι ανάμεσα στα [ ],
    τα στοιχεία των πινάκων τοποθετούνται κάθε φορά με απόσταση 2 στον κάθετο άξονα και 3 στον οριζόντιο
    το ένα από το άλλο. Με αυτά τα 2 νούμερα προκύπτουν και 2 τύποι, οι οποίοι βοηθάνε στην σωστή
    προσπέλαση των στοιχείων των πινάκων κατά την διάρκεια του παιχνιδιού, και στον υπολογισμό των
    διαστάσεων του παραθύρου που εμφανίζεται το πεδίο. Οι τύποι είναι οι εξής

    \begin{equation}
        x = rows + 2
    \end{equation}
    \begin{equation}
        y = columns \times 3 + 2
    \end{equation}

    Αφού τυπωθεί στην οθόνη ο πίνακας με κρυμμένα τα στοιχεία του, το οποίο είναι στην ουσία ένας 
    $Μ \times N$ πίνακας γεμισμένος με κενά, ξεκινάει το βασικό loop του παιχνιδιού, στο οποίο
    ο χρήστης μετακινείται από κελί σε κελί, επιλέγει την κίνηση που θέλει να κάνει πάνω σε κάθε κελί,
    και είτε χάνει ή νικάει. Προκειμένου να λειτουργήσει κάτι τέτοιο, μέσα στο loop γίνονται οι εξής
    λειτουργίες: Αρχικά ο κέρσορας μετακίνεται κάθε φορά που ο και χρήστης μετακινείται ώστε να μπορεί να δει 
    σε ποιο κελί βρίσκεται και ο χρήστης πρέπει έχει % επιλογές κλπ

    Λόγω του ότι το πρόγραμμα περιέχει πολλές μεταβλητές θεώρησα καλύτερο να εστιάσω
    στην λειτουργία του προγράμματος και όχι τόσο στο τι συμβολίζει η κάθε μεταβλητή.
    

\section{Διευκρινήσεις}



\section{Εργαλεία}

    \begin{itemize}
        \item Editors: Visual Studio Code, NVim
        \item Compiler: gcc
        \item Shell: zsh
        \item OS: Arch Linux
        \item Συγγραφή: \LaTeX
    \end{itemize}

\end{document}

