\documentclass{article}
\usepackage[utf8]{inputenc}
\usepackage[greek,english]{babel}
\usepackage{alphabeta}
\usepackage{fancyhdr}
\usepackage{listings}
\usepackage{xcolor}
 
% Settings

\definecolor{codered}{rgb}{0,0.6,0}
\definecolor{codegray}{rgb}{0.5,0.5,0.5}
\definecolor{codegreen}{rgb}{0.0,0.66,0.42}
\definecolor{hanblue}{rgb}{0.27,0.42,0.81}
\definecolor{codeorange}{rgb}{1.0,0.55,0.0}

\lstdefinestyle{mystyle}
{
    commentstyle=\color{codegray},
    otherkeywords=
    {
        >,<,.,;,-,!,=,~,&,*,+,-,
        int8_t, % Add more types
        printf, scanf, exit % Add more functions
    },
    keywordstyle=\color{hanblue},
    numberstyle=\tiny\color{codegray},
    stringstyle=\color{codegreen},
    basicstyle=\ttfamily\small,
    breakatwhitespace=false,         
    breaklines=true,                 
    captionpos=b,                    
    keepspaces=true,                 
    numbers=left,                    
    numbersep=5pt,                  
    showspaces=false,                
    showstringspaces=false,
    showtabs=false,                  
    tabsize=4
}

% Colors the digits
\lstset{literate=
   *{0}{{{\color{red!20!violet}0}}}1
    {1}{{{\color{red!20!violet}1}}}1
    {2}{{{\color{red!20!violet}2}}}1
    {3}{{{\color{red!20!violet}3}}}1
    {4}{{{\color{red!20!violet}4}}}1
    {5}{{{\color{red!20!violet}5}}}1
    {6}{{{\color{red!20!violet}6}}}1
    {7}{{{\color{red!20!violet}7}}}1
    {8}{{{\color{red!20!violet}8}}}1
    {9}{{{\color{red!20!violet}9}}}1
}
\lstset{style=mystyle}

\pagestyle{fancy}
\fancyhf{}

% Document starts here

\rhead{Εργασία 5}
\lhead{Πίνακες - Δείκτες - Αρχεία}
\rfoot{\thepage}

\title{Εργασία 5: Πίνακες - Δείκτες - Αρχεία}
\author{Χρήστος Μαργιώλης - Εργαστηριακό τμήμα 9}
\date{Ιανουάριος 2020}

\begin{document}

\begin{titlepage}
    \maketitle
\end{titlepage}

\renewcommand{\contentsname}{Περιεχόμενα}
\tableofcontents

\section{Δομή προγραμμάτων και οδηγίες εκτέλεσης}

    \subsection{Εκτέλεση από Linux}
\begin{lstlisting}[language=bash]
$ cd path-to-program
$ make
$ make run
$ make run ARGS=txt/data.txt #fcombinations ONLY
$ make clean #optional
\end{lstlisting}

    \subsection{Δομή φακέλων}


\section{combinations - συνδυασμοί}

    \subsection{main.c}
        \lstinputlisting[language=C]{../combinations/src/main.c}

    \subsection{combinations.c}
        \lstinputlisting[language=C]{../combinations/src/combinations.c}

    \subsection{combinations.h}
        \lstinputlisting[language=C]{../combinations/include/combinations.h}

    \subsection{arrhandler.c}
        \lstinputlisting[language=C]{../combinations/src/arrhandler.c}

    \subsection{arrhandler.h}
        \lstinputlisting[language=C]{../combinations/include/arrhandler.h}

    \subsection{Διάγραμμα ροής}

    \subsection{Περιγραφή υλοποιήσης}


\section{kcombinations - συνδυασμοί με K}

    \subsection{main.c}
        \lstinputlisting[language=C]{../kcombinations/src/main.c}

    \subsection{kcombinations.c}
        \lstinputlisting[language=C]{../kcombinations/src/kcombinations.c}

    \subsection{kcombinations.h}
        \lstinputlisting[language=C]{../kcombinations/include/kcombinations.h}

    \subsection{arrhandler.c}
        \lstinputlisting[language=C]{../kcombinations/src/arrhandler.c}

    \subsection{arrhandler.h}
        \lstinputlisting[language=C]{../kcombinations/include/arrhandler.h} 

    \subsection{Διάγραμμα ροής}

    \subsection{Περιγραφή υλοποιήσης}


\section{fcombinations - συνδυασμοί από αρχείο}

    \subsection{main.c}
        \lstinputlisting[language=C]{../fcombinations/src/main.c}

    \subsection{fcombinations.c}
        \lstinputlisting[language=C]{../fcombinations/src/fcombinations.c}

    \subsection{fcombinations.h}
        \lstinputlisting[language=C]{../fcombinations/include/fcombinations.h}

    \subsection{arrhandler.c}
        \lstinputlisting[language=C]{../fcombinations/src/arrhandler.c}

    \subsection{arrhandler.h}
        \lstinputlisting[language=C]{../fcombinations/include/arrhandler.h} 

    \subsection{Διάγραμμα ροής}

    \subsection{Περιγραφή υλοποιήσης}


\section{minesweeper - ναρκαλιευτής}

    \subsection{main.c}
        \lstinputlisting[language=C]{../minesweeper/src/main.c}

    \subsection{minesweeper.c}
        \lstinputlisting[language=C]{../minesweeper/src/minesweeper.c}

    \subsection{minesweeper.h}
        \lstinputlisting[language=C]{../minesweeper/include/minesweeper.h}

    \subsection{gameplay.c}
        \lstinputlisting[language=C]{../minesweeper/src/minesweeper.c}

    \subsection{gameplay.h}
        \lstinputlisting[language=C]{../minesweeper/include/gameplay.h}

    \subsection{settings.c}
        \lstinputlisting[language=C]{../minesweeper/src/settings.c}

    \subsection{settings.h}
        \lstinputlisting[language=C]{../minesweeper/include/settings.h}

    \subsection{outputs.c}
        \lstinputlisting[language=C]{../minesweeper/src/outputs.c}

    \subsection{outputs.h}
        \lstinputlisting[language=C]{../minesweeper/include/outputs.h}

    \subsection{Διάγραμμα ροής}

    \subsection{Περιγραφή υλοποιήσης}
    

\section{Διευκρινήσεις}



\section{Εργαλεία}

    \begin{itemize}
        \item Editors: Visual Studio Code, Vim
        \item Compiler: gcc
        \item Shell: zsh
        \item OS: Arch Linux
        \item Συγγραφή: \LaTeX
    \end{itemize}

\end{document}

