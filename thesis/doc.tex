\documentclass[12pt]{article}
\usepackage[utf8]{inputenc}
\usepackage[greek,english]{babel}
\usepackage{alphabeta}
\usepackage{fancyhdr}
\usepackage{listings}
\usepackage{mathtools}
\usepackage{xcolor}
\usepackage{float}
\usepackage{siunitx}
\usepackage[nottoc,numbib]{tocbibind}
\usepackage[margin=0.5in]{geometry}
\usepackage[backend=bibtex]{biblatex}

\lstset {
        basicstyle=\ttfamily,
        columns=fullflexible,
        breaklines=true,
        keepspaces=true,
	showstringspaces=false
}

\title{Διπλωματική Εργασία
\linebreak
\linebreak
Μελέτη και ανάπτυξη τεχνικών για την παρακολούθηση και την αποσφαλμάτωση της
εκτέλεσης εντολών σε υπολογιστικά συστήματα}
\author{Χρήστος Μαργιώλης \\ Α.Μ. 19390133}
\date{}

\begin{document}

\begin{titlepage}
        \maketitle
        \begin{figure}[t!]
        \begin{center}
        \includegraphics[scale=0.3]{./res/uniwalogo.png} \\
        \Large
	\textbf{Πανεπιστήμιο Δυτικής Αττικής} \\
        \large
	Σχολή Μηχανικών \\
        Τμήμα Μηχανικών Πληροφορικής και Υπολογιστών \\
        \end{center}
        \end{figure}
        \begin{center}
	Εισηγητής: Παναγιώτης Καρκαζής % TODO βαθμίδα
        \end{center}
\end{titlepage}

\shipout\null

\begin{center}
Διπλωματική Εργασία
\linebreak
\linebreak
Μελέτη και ανάπτυξη τεχνικών για την παρακολούθηση και την αποσφαλμάτωση της
εκτέλεσης εντολών σε υπολογιστικά συστήματα
\linebreak
\linebreak
Χρήστος Μαργιώλης
\linebreak
Α.Μ. 19390133
\end{center}

Εισηγητής:

\begin{center}
Παναγιώτης Καρκαζής % TODO βαθμίδα
\end{center}

Εξεταστική επιτροπή:

\begin{center}
FILLME % TODO
\linebreak
\linebreak
Ημερομηνία εξέτασης: FILLME % TODO
\end{center}

\pagebreak
\shipout\null

\begin{center}
\textbf{Δήλωση συγγραφέα διπλωματικής εργασίας}
\end{center}

% TODO

\begin{center}
Ο Δηλών

% TODO υπογραφή
\end{center}

\pagebreak
\shipout\null

\begin{center}
\textbf{Ευχαριστίες}
\end{center}

% TODO

\pagebreak
\shipout\null

\begin{center}
\textbf{Περίληψη}
\end{center}

Η εργασία αποσκοπεί στην μελέτη τεχνικών που χρησιμοποιούνται στην ανάλυση και
αποσφαλμάτωση λογισμικού μέσω της καταγραφής και παρακολούθησης των εντολών που
εκτελούνται σε ένα επεξεργαστή. Στο πλαίσιο της διπλωματικής θα σχεδιαστεί και
θα αναπτυχθεί επέκταση του εργαλείου DTrace η οποία θα παρέχει την δυνατότητα
παρακολούθησης οποιασδήποτε μεμονωμένης εντολής assembly εντός μιας δεδομένης
συνάρτησης του πυρήνα του λειτουργικού συστήματος FreeBSD.
% TODO improve

\begin{center}
\textbf{Abstract}
\end{center}

% TODO

\pagebreak

\renewcommand{\contentsname}{Περιεχόμενα}
\tableofcontents
\pagebreak

\section{Εισαγωγή}

% TODO

\renewcommand{\refname}{Βιβλιογραφία}
\begin{thebibliography}{00}
\end{thebibliography}

\end{document}
