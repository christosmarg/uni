\documentclass[12pt]{article}
\usepackage[utf8]{inputenc}
\usepackage[greek,english]{babel}
\usepackage{cite}
\usepackage{alphabeta}
\usepackage{fancyhdr}
\usepackage{listings}
\usepackage{float}
\usepackage{hyperref}
\usepackage{tikz}
\usetikzlibrary{shapes.geometric, shapes.multipart, arrows}
\usepackage[margin=0.5in]{geometry}
\usepackage[nottoc,numbib]{tocbibind}
%\usepackage[backend=bibtex]{biblatex}
\def\BibTeX{{\rm B\kern-.05em{\sc i\kern-.025em b}\kern-.08em
    T\kern-.1667em\lower.7ex\hbox{E}\kern-.125emX}}

% XXX https://polynoe.lib.uniwa.gr/xmlui/handle/11400/55
% XXX https://polynoe.lib.uniwa.gr/xmlui/bitstream/handle/11400/8820/Vangelis_18390008.pdf?sequence=1&isAllowed=y

\lstset {
        basicstyle=\ttfamily,
        columns=fullflexible,
        breaklines=true,
        keepspaces=true,
	showstringspaces=false
}

\tikzstyle{header} = [rectangle, minimum width=3cm, minimum
	height=1cm, text centered, text width=3cm]
\tikzstyle{box} = [rectangle, minimum width=3cm, minimum
	height=1cm, text centered, text width=3cm, draw=black]
\tikzstyle{instr} = [trapezium, trapezium left angle=70, trapezium right
	angle=110, minimum width=3cm, minimum height=1cm, text centered,
	draw=black]
\tikzstyle{func} = [rectangle, rounded corners, minimum width=3cm, minimum
	height=1cm, text centered, text width=3cm, draw=black]
\tikzstyle{decision} = [diamond, minimum width=3cm, minimum height=1cm, text
	centered, draw=black]
\tikzstyle{arrow} = [thick,->,>=stealth, rounded corners]
\tikzstyle{chain} = [thick,-,>=stealth]

\title{Διπλωματική Εργασία
\linebreak
\linebreak
Μελέτη και ανάπτυξη τεχνικών για την παρακολούθηση και την αποσφαλμάτωση της
εκτέλεσης εντολών σε υπολογιστικά συστήματα}
\author{Χρήστος Μαργιώλης \\ Α.Μ. 19390133}
\date{}

\begin{document}

\begin{titlepage}
        \clearpage\maketitle
	\thispagestyle{empty}
        \begin{figure}[t!]
        \begin{center}
        \includegraphics[scale=0.3]{./res/uniwalogo.png} \\
        \Large
	\textbf{Πανεπιστήμιο Δυτικής Αττικής} \\
        \large
	Σχολή Μηχανικών \\
        Τμήμα Μηχανικών Πληροφορικής και Υπολογιστών \\
        \end{center}
        \end{figure}
        \begin{center}
	Εισηγητής: Παναγιώτης Καρκαζής
        \end{center}
\end{titlepage}

% TODO number page
\shipout\null

% TODO make headers bigger

\begin{center}
Διπλωματική Εργασία
\linebreak
\linebreak
Μελέτη και ανάπτυξη τεχνικών για την παρακολούθηση και την αποσφαλμάτωση της
εκτέλεσης εντολών σε υπολογιστικά συστήματα
\linebreak
\linebreak
Χρήστος Μαργιώλης
\linebreak
Α.Μ. 19390133
\end{center}

Εισηγητής:

\begin{center}
Παναγιώτης Καρκαζής, <++> % TODO βαθμίδα
\end{center}

Εξεταστική επιτροπή:

\begin{center}
<++>
\linebreak
\linebreak
Ημερομηνία εξέτασης: <++>
\end{center}

\pagebreak
\shipout\null

\begin{center}
\textbf{Δήλωση συγγραφέα διπλωματικής εργασίας}
\end{center}

<++>

\begin{center}
Ο Δηλών

<++> % TODO υπογραφή
\end{center}

\pagebreak
\shipout\null

\begin{center}
\textbf{Ευχαριστίες}
\end{center}

% TODO
% Mark, Mitchell, καθηγητής, οικογένεια, FBSDF, ...

<++>

\pagebreak
\shipout\null

\begin{center}
\textbf{Περίληψη}
\end{center}

Η εργασία αποσκοπεί στην μελέτη τεχνικών που χρησιμοποιούνται στην ανάλυση και
αποσφαλμάτωση λογισμικού μέσω της καταγραφής και παρακολούθησης των εντολών που
εκτελούνται σε ένα επεξεργαστή. Στο πλαίσιο της διπλωματικής θα σχεδιαστεί και
θα αναπτυχθεί επέκταση του εργαλείου DTrace η οποία θα παρέχει την δυνατότητα
παρακολούθησης οποιασδήποτε μεμονωμένης εντολής assembly εντός μιας δεδομένης
συνάρτησης του πυρήνα του λειτουργικού συστήματος FreeBSD.
% TODO improve: mention inline?

<++> % TODO keywords

\begin{center}
\textbf{Abstract}
\end{center}

<++>

\pagebreak

\renewcommand{\contentsname}{Περιεχόμενα}
\tableofcontents
\pagebreak

\section{Συντομογραφίες}

<++>

\section{Εισαγωγή στην παρακολούθηση (tracing)}

Με τον όρο tracing εννοούμε την παρακολούθηση της ροής εκτέλεσης ενός
προγράμματος σε πραγματικό χρόνο (real-time), με σκοπό την εξαγωγή δεδομένων,
που είναι χρήσιμα κυρίως στην αποσφαλμάτωση (debugging) του προγράμματος, στην
καλύτερη κατανόση της ροής εκτέλεσης, καθώς και στην εύρεση τυχόν σημείων που
προκαλούν καθυστερήσεις (bottleneck). Κατά μία έννοια το tracing μπορεί να
θεωρηθεί ως μία καλύτερη και πιο δυναμική μορφή καταγραφής πληροφοριών
(logging), διότι δίνεται η δυνατότητα στον χρήστη να εξάγει αφηρημένες
πληροφορίες από ένα πρόγραμμα, χωρίς να χρειάζεται αναγκαστικά η συγγραφή
περαιτέρω πηγαίου κώδικα μέσα στο πρόγραμμα και η επαναμεταγλώττισή του, κάτι
το οποίο, στην περίπτωση μεγάλων προγραμμάτων όπως ο πυρήνας ενός λετουργικού
συστήματος, τείνει να είναι μία χρονοβόρα διαδικασία.

% TODO
% «παρακολούθηση ροής εκτέλεσης»?
% what can we do with tracing? (func timing, args, stack trace, ...)
% tracing vs debugging
% arbitrary questions that diagnostic programs with pre-defined (canned) questions cannot always answer
%	top(1), nestat(1), procstat(1), vmstat(8), systat(1)
%	all these static tools are fundamentally limited
%	they ask pre-defined questions and answer them in pre-defined formats
%	they are also not realtime
% realtime
% used in productions systems, mostly by sysadmins and systems/os developers
% history: tamches paper
% REFERENCES

\section{Εισαγωγή στο DTrace}

% TODO
% https://www.illumos.org/books/dtrace/
% https://www.youtube.com/watch?v=E06GVdH-LX0
% https://www.youtube.com/watch?v=ctTyYlDbAIM
% gsoc emails with mark
% See paper, slides, articles
% https://www.bsdcan.org/2017/schedule/attachments/432_dtrace_internals.pdf
%
% origins: illumos -> freebsd
% performance overhead from dtrace
% kernel module
%	what is a kmod
%	how do you load kmods?
% Userspace dtrace
% REFERENCES

<++>

\subsection{Providers}

% TODO
% Providers
%	mark 00:29:00
%	fbt
%		entry and return points of a func (with args) only
%		Breakpoint instructions
%		caveats: tail calls, inline funcs
%		pid -> fbt for userspace
%			00:57:00

<++>

\subsubsection{FBT} \label{fbt}

<++>

\subsection{Probes}

% TODO
%		name: the probe itself: point of instrumentation (entry, return, instruction)
%			"piece of kernel code that gets executed" -> piece of code we care about
%			when the probe is enabled, dtrace is called when the code hits
%			that point, does whatever actions we asked it to do, and then
%			hands execution back to the program.

<++>

\subsection{Γλώσσα D}

% TODO
% D language
%	link to docs
%	dtrace handbook
%	Kind of like C + awk
%	safe to execute (i.e., cannot crash). gives you a limited but powerful
%	set of tools so that crashing the kernel or a program is not possible
%	post-processing: data aggregation, graphs, histograms
%	aggregations
%		collect data and summarize them later
%		printed when dtrace(1) exits (or SIGINFO -> ctrl+t)
%		mark 00:39:00-00:54:00

<++>

\subsubsection{Δομή εντολών DTrace}

% TODO
%	provider:module:function:name /predicate/ {actions}
%		GIVE EXAMPLES
%
%		provider: set of logically related probes
%			kind of like a namespace for probes: syscall, sched, ip, fbt, ...
%		module: explain
%		function: roughly corresponds to a C function inside the program we're tracing
%			mark 00:17:00
%		name: the probe itself: point of instrumentation (entry, return, instruction)
%			"piece of kernel code that gets executed" -> piece of code we care about
%			when the probe is enabled, dtrace is called when the code hits
%			that point, does whatever actions we asked it to do, and then
%			hands execution back to the program.
%
%			probes fire when an event happens (e.g., io, signal, ...)
%		predicate (optional): quick filters, like awk
%		action (optional): the probe body: data collection, aggregation, printing, processing
%			mark 00:35:00
%
%		module, function, name can take wildcarding

Οι εντολές DTrace ορίζονται ως εξής:

\begin{lstlisting}
	<provider>:<module>:<function>:<name> /<predicate>/ {<actions>}
\end{lstlisting}

<++>

\subsubsection{Global μεταβλητές}

% TODO
%	global variables
%		pid, execname, timestamp (nanosecs)
%		they are context-dependent, i.e., different for every thread

<++>

\subsubsection{CTF}

% TODO
%	CTF
%	link to docs
%	diagrams
%		dwarf
%			gdb (and inline tracing) uses it to unwind stack, find
%			function names, arguments, types, variable locations,
%			
%			dwarf metadata is generated at compile time
%
%			dwarfdump
%		ctf is a simplified representation of C types so that we can
%		fetch them in D
%			ctfdump -t /boot/kernel/kernel | less
%			probe arguments are defined using ctf, roughly like C types
%				dtrace_ip.4

<++>

\subsubsection{DIF}

% TODO
%	Compilation: DIF (D intermediary format)
%		link to docs
%		https://www.usenix.org/legacy/event/usenix04/tech/general/full_papers/cantrill/cantrill_html/
%		diagram
%		dtrace -S -> dump bytecode for a given d script

Για να εισαχθεί ένα D script στον πυρήνα του λειτουργικού συστήματος,
χρησιμοποιείται η κωδικοποίηση DIF (D Intermediary Format). Η κωδικοποίηση αυτή
έχει ως στόχο την μεταγλώττιση του D script σε μία ενδιάμεση byte κωδικοποίηση
(byte code), η οποία μπορεί να εκτελεστεί από το DTrace εντός του kernel. Αφού
το script έχει μεταγλωττιστεί, φορτώνεται στο kernel από το DTrace, και
εκτελείται για τα probes που έχουν οριστεί όταν αυτά ενεργοποιηθούν.

Αντίστοιχα στο Linux, χρησιμοποείται η κωδικοποίηση \lstinline{eBPF}.

<++>

\subsection{Επικοινωνία δεδομένων μεταξύ kernel και userspace}

% TODO
% Buffers
%	https://www.illumos.org/books/dtrace/chp-buf.html#chp-buf
%	what parameters can be modified? do they introduce overhead (switchrate
%	yes)?
%	diagrams
% ioctls?

Η επικονωνία δεδομένων μεταξύ kernel και userspace επιτυχάνεται μέσω ενός
ζευγαριού buffers, διαφορετικό για κάθε πυρήνα (per-CPU). Aνά πάσα στιγμή ο
ένας buffer είναι ανενεργός και ο άλλος ενεργός, και κάθε ένα δευτερόλεπτο, οι
buffers ανταλλάσονται, οπότε αυτός που ήταν ενεργός τώρα είναι ανενεργός και
αυτός που ήταν ανενεργός τώρα είναι ενεργός. Στον ενεργό buffer γράφονται τα
δεδομένα από το kernel τα οποία οποία με την ανταλλαγή των buffers στέλνονται
και γίνονται διαθέσιμα προς διάβασμα από το userspace μέσω του πλέον ανενεργού
buffer. Με άλλα λόγια, το kernel γράφει στον ενεργό buffer, και το userspace
διαβάζει από τον ανενεργό. Αυτό έχει ως αποτέλεσμα την συνεχή ροή δεδομένων από
το kernel προς το userspace, διότι ταυτόχρονα γίνονται διάβασμα και καταγραφή
δεδομένων.

<++>

\subsection{Scripts υψηλότερου επιπέδου}

% TODO
% higher-level scripts
%	lockstat
%	flamegraphs
%	dwatch

<++>

\subsection{Παραδείγματα}

% TODO
% examples
%	mark 00:30:00
%	mark 00:36:00
%	mark 00:39:00-00:54:00
%	/usr/share/examples/dtrace
%	flamegraphs
%		mark 01:11:00
%	writing a provider

<++>

\section{Inline tracing}

% TODO
% See paper, slides, articles, diagrams
% Μήπως να το ενσωματώσω σε κάποια άλλη ενότητα;

<++>

\subsection{Τι είναι inline συναρτήσεις}

<++>

\subsection{Γιατί είναι δύσκολη η παρακολούθηση inline συναρτήσεων}

% TODO
% See paper, slides, articles, links to docs

<++>

\section{kinst}

% TODO
% See paper, slides, articles, commits, manpage
%	https://reviews.freebsd.org/D40874
% Overview
% Architecture dependent parts
%	emulation
% Inline tracing
%	Τι είναι inline functions
%	Γιατί είναι δύσκολο να τις κάνουμε trace
%	Πως το λύνει αυτό ο kinst
%	dwarf
%	elf
%	inlinecall, omitrbp
% diagrams
% tail-calls?
%
%kinst creates probes on-demand, meaning it searches for and parses the
%function's instructions each time dtrace(1) is run, and not at module
%load time.  This is in contrast to FBT's load-time parsing, since kinst
%can potentially create thousands of probes for just a single function,
%instead of up to two (entry and return) in the case of FBT.  A result of
%this is that dtrace -l -P kinst will not match any probes.


Ο \lstinline{kinst} είναι ένας νέος DTrace provider, ο οποίος αρχικά
δημιουργήθηκε με στόχο να αντιμετωπίσει τους περιορισμούς του FBT (βλέπε
ενότητα \ref{fbt}), δηλαδή την έλλειψη δυνατότητας παρακολούθησης inline
συναρτήσεων, καθώς και γενικότερα της πιο εξειδικευμένής παρακολουθήσης πέραν
της αρχής ή/και του τέλους μίας συνάρτησης.

Στον πηγαίο κώδικα του FreeBSD, ο κώδικας του \lstinline{kinst} βρίσκεται στο
\lstinline{sys/cddl/dev/kinst} \cite{kinstsrc} και υποστηρίζεται στις
αρχιτεκτονικές x86-64 (AMD64), ARM64 και RISC-V.

Το όνομα "kinst" είναι εμπνευσμένο από την δημοσίευση των Tamches \& Miller
\cite{tamches}, στην οποία κάνουν αναφορά στο πειραματικό εργαλείο
παρακολούθησης που ανέπτυξαν, ως "KernInst".

<++>

\subsection{Χρήση}

Ο \lstinline{kinst} δέχεται τις παρακάτω τρεις συντάξεις:

\begin{itemize}
	\item \lstinline{kinst::<function>:}
	\item \lstinline{kinst::<function>:<instruction>}
	\item \lstinline{kinst::<inline_function>:<entry|return>}
\end{itemize}

Με την πρώτη σύνταξη παρακολουθούνται \textit{όλες} οι εντολές assembly της
συνάρτησης \lstinline{function}. Για παράδειγμα:

<++>

Με την δεύτερη σύνταξη παρακολουθείται εντός της συνάρτησης
\lstinline{function} μόνο η εντολή που βρίσκεται στο offset που ορίστηκε στο
πεδίο \lstinline{instruction}. Για παράδειγμα:

<++>

Με την τρίτη σύνταξη παρακολουθείται η αρχή (\lstinline{entry}) ή/και το τέλος
(\lstinline{return}) μίας inline συνάρτησης \lstinline{inline_function}. Για
παράδειγμα:

\begin{lstlisting}[caption={Inline tracing}, captionpos=b]
	# dtrace -n 'kinst::critical_enter:return'
	dtrace: description 'kinst::critical_enter:return' matched 130 probes
	CPU     ID                    FUNCTION:NAME
	  1  71024                spinlock_enter:53
	  0  71024                spinlock_enter:53
	  1  70992                uma_zalloc_arg:49
	  1  70925    malloc_type_zone_allocated:21
	  1  70994                uma_zfree_arg:365
	  1  70924             malloc_type_freed:21
	  1  71024                spinlock_enter:53
	  0  71024                spinlock_enter:53
	  0  70947         _epoch_enter_preempt:122
	  0  70949           _epoch_exit_preempt:28
	  ^C
\end{lstlisting}

<++>

\subsection{Ενορχήστρωση εντολών assembly}

% TODO
%Ο \lstinline{kinst} διαθέτει ένα αρχείο συσκευής στο
%\lstinline{/dev/dtrace/kinst}, το οποίο χρησιμοποιείται για <++>

<++>

\subsection{«Τραμπολίνο»}

% TODO improve wording

Τραμπολίνο (trampoline) ονομάζεται ένα \textit{εκτελέσιμο} κομμάτι μνήμης, το
οποίο χρησιμοποιείται ως περιοχή αναπήδησης (jump) για την εκτέλεση κάποιας
εντολής και επιστροφής στην κανονική ροή του προγράμματος.

Ο \lstinline{kinst} κάνει χρήση τραμπολίνων για την εκτέλεση των εντολών των
οποίων παρακολουθεί, σε αντίθεση με τον FBT (\ref{fbt}), ο οποίος τις
«μιμείται»/υλοποιεί (emulation). Ωστόσο, μία βασική διαφορά του FBT είναι ότι
εφόσον έχει την δυνατότητα να παρακολουθήσει μόνο την αρχή και το τέλος μίας
συνάρτησης, τα οποία σημεία ορίζονται με συγκεκριμένες εντολές assembly για
κάθε αρχιτεκτονική
\footnote{
	Για παράδειγμα, στην αρχιτεκτονική x86-64, η εντολή \lstinline{push
	\%rbp} δηλώνει το ξεκίνημα του \textit{προλόγου} μίας συνάρτησης, και η
	εντολή \lstinline{ret} το τέλος του \textit{επιλόγου}. % TODO cite
},
το σύνολο των εντολών που πρέπει να υλοποίησει ο FBT είναι μικρό, καθώς και οι
εντολές απλές, με αποτέλεσμα να μην χρειάζεται η επιπλέον πολυπλοκότητα που
προσθέτει το τραμπολίνο. Ο \lstinline{kinst} όμως μπορεί να παρακολουθήσει εν
δυνάμει (σχεδόν) όλες τις εντολές assembly για κάθε αρχιτεκτονική που
υποστηρίζει, οπότε είναι προφανές πως η προσέγγιση της υλοποίησης εντολών, δεν
μπορεί να λειτουργήσει σε αυτήν την περίπτωση. Παρ' όλα αυτά, υπάρχουν μερικές
εξαιρέσεις που θα συζητηθούν παρακάτω. % TODO ref to emulation

Η χρήση του τραμπολίνου γίνεται ως εξής. <++>

\subsubsection{Διάταξη}

<++>

\subsubsection{Σημειώσεις υλοποίησης για x86-64}

<++>

\subsubsection{Σημειώσεις υλοποίησης για ARM64 και RISC-V}

<++>

\subsection{Παρακολούθηση inline συναρτήσεων}

% TODO
% https://margiolis.net/w/kinst_inline/

<++>

\subsubsection{Πρότυπο DWARF}

% TODO
% https://margiolis.net/w/dwarf_inline/

<++>

\subsubsection{Πρότυπο ELF}

% TODO
% Bib
%	https://wiki.osdev.org/ELF

<++>

\subsubsection{Υπολογισμός ορίων κλήσεων}

<++>

\subsubsection{Εύρεση καλούσας συνάρτησης}

<++>

\subsubsection{Υπολογισμός των offsets \lstinline{entry} και \lstinline{return}}

<++>

\section{Πειράματα}

% TODO
% Περιβάλλον εκτέλεσης
% See paper, slides, articles

<++>

\section{Συμπεράσματα}

<++>

% TODO
% Sort and cleanup

\pagebreak
\renewcommand{\refname}{Βιβλιογραφία}
\begin{thebibliography}{00}
\bibitem{dtracebook} Illumos Operating System ``Dynamic Tracing Guide''.
	\href{https://illumos.org/books/dtrace}{https://illumos.org/books/dtrace} 
\bibitem{kinstsrc} FreeBSD src ``kinst''
	\href{https://cgit.freebsd.org/src/tree/sys/cddl/dev/kinst}
	{https://cgit.freebsd.org/src/tree/sys/cddl/dev/kinst}
\bibitem{tamches} Tamches, Ariel \& Miller, Barton P. ``Fine-Grained Dynamic
	Instrumentation of Commodity Operating System Kernels''.
	\href{https://www.usenix.org/legacy/publications/library/proceedings/osdi99/full\_papers/tamches/tamches.pdf}
	{https://www.usenix.org/legacy/publications/library/proceedings/osdi99/full\_papers/tamches/tamches.pdf}
\bibitem{dwarfstd} The DWARF Standard.
	\href{https://dwarfstd.org/}{https://dwarfstd.org/}
\bibitem{margdwarf} Christos Margiolis ``Using DWARF to find call sites of
	inline functions``.
	\href{https://margiolis.net/w/dwarf\_inline/}{https://margiolis.net/w/dwarf\_inline/}
\bibitem{kinstinline} Christos Margiolis ``Inline function tracing with the
	kinst DTrace provider''.
	\href{https://margiolis.net/w/kinst\_inline/}{https://margiolis.net/w/kinst\_inline/}
\bibitem{inlinecall} Sourcehut, inlinecall(1).
	\href{https://github.com/christosmarg/inlinecall}{https://github.com/christosmarg/inlinecall}
\bibitem{markpres} Mark Johnston ``Introduction to DTrace on FreeBSD''.
	\href{https://www.youtube.com/watch?v=E06GVdH-LX0}{https://www.youtube.com/watch?v=E06GVdH-LX0}
\end{thebibliography}

\pagebreak
\section{Παράρτημα}


% TODO
% Να αδειάσω τον κώδικα όπως είναι ή να βάλω link;

<++>

\end{document}
