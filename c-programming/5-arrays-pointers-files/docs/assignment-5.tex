\documentclass{article}
\usepackage[utf8]{inputenc}
\usepackage[greek,english]{babel}
\usepackage{alphabeta}
\usepackage{fancyhdr}
\usepackage{listings}
\usepackage{mathtools}
\usepackage{xcolor}
 
% Settings

\definecolor{codered}{rgb}{0,0.6,0}
\definecolor{codegray}{rgb}{0.5,0.5,0.5}
\definecolor{codegreen}{rgb}{0.0,0.66,0.42}
\definecolor{hanblue}{rgb}{0.27,0.42,0.81}
\definecolor{codeorange}{rgb}{1.0,0.55,0.0}

\lstdefinestyle{mystyle}
{
    commentstyle=\color{codegray},
    otherkeywords=
    {
        >,<,.,;,-,!,=,~,&,*,+,-,
        int8_t, % Add more types
        printf, scanf, exit % Add more functions
    },
    keywordstyle=\color{hanblue},
    numberstyle=\tiny\color{codegray},
    stringstyle=\color{codegreen},
    basicstyle=\ttfamily\small,
    breakatwhitespace=false,         
    breaklines=true,                 
    captionpos=b,                    
    keepspaces=true,                 
    numbers=left,                    
    numbersep=5pt,                  
    showspaces=false,                
    showstringspaces=false,
    showtabs=false,                  
    tabsize=4
}

% Colors the digits
\lstset{literate=
   *{0}{{{\color{red!20!violet}0}}}1
    {1}{{{\color{red!20!violet}1}}}1
    {2}{{{\color{red!20!violet}2}}}1
    {3}{{{\color{red!20!violet}3}}}1
    {4}{{{\color{red!20!violet}4}}}1
    {5}{{{\color{red!20!violet}5}}}1
    {6}{{{\color{red!20!violet}6}}}1
    {7}{{{\color{red!20!violet}7}}}1
    {8}{{{\color{red!20!violet}8}}}1
    {9}{{{\color{red!20!violet}9}}}1
}
\lstset{style=mystyle}

\pagestyle{fancy}
\fancyhf{}

% Document starts here

\rhead{Εργασία 5}
\lhead{Πίνακες - Δείκτες - Αρχεία}
\rfoot{\thepage}

\title{Εργασία 5: Πίνακες - Δείκτες - Αρχεία}
\author{Χρήστος Μαργιώλης - Εργαστηριακό τμήμα 9}
\date{Ιανουάριος 2020}

\begin{document}

\begin{titlepage}
    \maketitle
\end{titlepage}

\renewcommand{\contentsname}{Περιεχόμενα}
\tableofcontents

\section{Δομή προγραμμάτων και οδηγίες εκτέλεσης}

    \subsection{Εκτέλεση από Unix/Linux/Mac}

    Λόγω του ότι ορισμένες βιβλιοθήκες που έχω χρησιμοποιήσει δεν είναι συμβατές με τα Windows, τα προγράμματα,
    και \textit{κυρίως} ο ναρκαλιευτής, είναι περιορισμένα για συστήματα Unix.

\begin{lstlisting}[language=bash]
$ cd path-to-program
$ make
$ make run
$ make run ARGS=txt/data40.txt #fcombinations MONO
$ make clean
\end{lstlisting}
    
    Προκειμένου να εκτελεστεί ο ναρκαλιευτής χρειάζονται τα παρακάτω dependencies:  
    \begin{itemize}
        \item cmake
        \item ncurses
        \item SDL2
        \item SDL2-mixer
    \end{itemize}

    \subsection{Δομή φακέλων}

    Το κάθε πρόγραμμα, είναι δομημένο ως εξής: Υπάρχουν πέντε \textit{βασικοί} φάκελοι, καθώς και ένα Makefile
    στο top directory. Στον φάκελο src βρίσκονται οι πηγαίοι κώδικες, στον include τα header
    files, στον οbj τα object files και στον bin το εκτελέσιμο αρχείο. Στον φάκελο txt
    υπάρχουν τα text files που διαβάζονται οι γράφονται από το κάθε πρόγραμμα.
    Το Makefile είναι υπεύθυνο για την μεταγλώττιση όλων των αρχείων μαζί, και την τοποθέτησή
    τους στους κατάλληλους φακέλους, την εκτέλεση των προγραμμάτων, καθώς και τον καθαρισμό των
    φακέλων (διαγράφει τα object files και το εκτελέσιμο με την εντολή make clean).

\section{combinations - συνδυασμοί}

    \subsection{main.c}
        \lstinputlisting[language=C]{../combinations/src/main.c}

    \subsection{combinations.c}
        \lstinputlisting[language=C]{../combinations/src/combinations.c}

    \subsection{combinations.h}
        \lstinputlisting[language=C]{../combinations/include/combinations.h}

    \subsection{arrhandler.c}
        \lstinputlisting[language=C]{../combinations/src/arrhandler.c}

    \subsection{arrhandler.h}
        \lstinputlisting[language=C]{../combinations/include/arrhandler.h}

    \subsection{Περιγραφή υλοποιήσης}

    Το πρόγραμμα αυτό, όπως και τα ακόλουθα 2 προγράμματα, έχουν ως στόχο την εμφάνιση
    συνδυασμών, εφόσον πληρούν τις προϋποθέσεις του φύλλου εργασίας. Όσο αφορά το πρόγραμμα όμως,
    στο main.c αποθηκεύεται η τιμή του Ν, ο πίνακας Ν θέσεων που έχει τα στοιχεία που δίνει ο χρήστης,
    καθώς και τα ζευγάρια $x_{1}$, $x_{2}$, $y_{1}$ και $y_{2}$. Από την main, τέλος, καλείται και η
    συνάρτηση εκτύπωσης των συνδυασμών, από την οποία καλούνται επιπλέον συναρτήσεις προκειμένου να εκτελεστεί
    σωστά αυτή η λειτουργία.  
    
    Στο arrhandler.c γίνεται όλος ο χειρισμός του πίνακα - δηλαδή, η \textit{δυναμική} του δέσμευση,
    το γέμισμα του, στο όποιο εκτελούνται συναρτήσεις για τον έλεγχο του αν ένας αριθμός κατά την εισαγωγή τιμών
    έχει ήδη εισαχθεί, και αν τηρούνται τα όρια $[1, 49]$. Επίσης, γίνεται ταξινόμηση του πίνακα με τον αλγόριθμο
    quicksort, και τέλος, μία επιπλέον λειτουργία, που θα αναλυθεί στην συνέχεια.  

    Στο combinations.c, γίνονται όλες οι λειτουργίες που αφορούν τους συνδυασμούς, τα ζευγάρια $x$ και $y$, και τους
    ελέγχους για το αν τηρούνται οι εξής προϋποθέσεις προκειμένου να εμφανιστεί ένας συνδυασμός 6 αριθμών.
    
    \begin{itemize}
        \item Το πλήθος των αρτίων αριθμών του συνδυασμού να βρίσκεται στο διάστημα $[x_{1}, x_{2}]$
        \item Το άθροισμα των έξι αρθιμών του συνδυασμού να βρίσκεται στο διάστημα $[y_{1}, y_{2}]$
    \end{itemize}

    Έπειτα, μετριούνται πόσοι συνδυασμοί δεν πληρούσαν τον πρώτο όρο ή μόνο τον δεύτερο όρο, ποιοί συνδυασμοί τυπώθηκαν,
    καθώς και την συχνότητα εμφάνισης του κάθε στοιχείου στο σύνολο των συνδυασμών που τυπώθηκαν.
    Για την τελευταία λειτουργία, χρησιμοποιείται η συναρτήση findpos() που βρίσκεται στο arrhandler.c, και ελέγχει για κάθε
    στοιχείο του τρέχοντως συνδυασμού, ποια θέση έχει στον πίνακα Ν θέσεων, η οποία αντιστοιχίζεται με έναν τρίτο πίνακα,
    τον freqArr, στον οποίο αποθηκεύονται ως ακέραιες τιμές η συνχότητες εμφάνισης του κάθε στοιχείου.
    Ο λόγος που χρειάζεται μια συνάρτηση που βρίσκει την τοποθεσία του κάθε στοιχείου στον πίνακα Ν θέσεων είναι ώστε να αυξηθεί κατά 
    ένα κάθε φορά ο πίνακας των συχνοτήτων στην κατάλληλη θέση, η οποία όπως προανέφερα, αντιστοιχεί στο στοιχείο του πίνακα Ν θέσεων, 
    δηλαδή στο στοιχείο που έλεγχεται κάθε φορά κατα την προσπέλαση του πίνακα του τρέχοντως συνδυασμού.

\section{kcombinations - συνδυασμοί με K}

    \subsection{main.c}
        \lstinputlisting[language=C]{../kcombinations/src/main.c}

    \subsection{kcombinations.c}
        \lstinputlisting[language=C]{../kcombinations/src/kcombinations.c}

    \subsection{kcombinations.h}
        \lstinputlisting[language=C]{../kcombinations/include/kcombinations.h}

    \subsection{arrhandler.c}
        \lstinputlisting[language=C]{../kcombinations/src/arrhandler.c}

    \subsection{arrhandler.h}
        \lstinputlisting[language=C]{../kcombinations/include/arrhandler.h} 

    \subsection{Περιγραφή υλοποιήσης}

    Όπως και στο προηγούμενο πρόγραμμα, το combinations, το kcombinations λειτουργεί με τον ίδιο ακριβώς τρόπο,
    με την διαφορά οτι η σταθερά COMBSN = 6 του προηγούμενου προγράμματος έχει αντικατασταθεί με την μεταβλητή Κ,
    η οποία δίνεται κάθε φορά από τον χρήστη.

\section{fcombinations - συνδυασμοί από αρχείο}

    \subsection{main.c}
        \lstinputlisting[language=C]{../fcombinations/src/main.c}

    \subsection{fcombinations.c}
        \lstinputlisting[language=C]{../fcombinations/src/fcombinations.c}

    \subsection{fcombinations.h}
        \lstinputlisting[language=C]{../fcombinations/include/fcombinations.h}

    \subsection{arrhandler.c}
        \lstinputlisting[language=C]{../fcombinations/src/arrhandler.c}

    \subsection{arrhandler.h}
        \lstinputlisting[language=C]{../fcombinations/include/arrhandler.h} 

    \subsection{Περιγραφή υλοποιήσης}

    Το πρόγραμμα αυτό, επίσης όπως και τα δύο προηγούμενα προγράμματα, λειτουργεί με την ίδια λογική
    ακριβώς, με την διαφορά οτι όλα τα δεδομένα διαβάζονται από αρχείο, το οποίο σημαίνει ότι σε αυτό το
    πρόγραμμα δεν υπάρχουν όλες οι printf() - scanf() που ζητάνε από τον χρήστη να εισάγει δεδομένα, εφόσον έχουν
    αντικατασταθεί από την fscanf() η οποία θα διαβάσει τα δεδομένα από αρχείο. 

\section{minesweeper - ναρκαλιευτής}

    \subsection{main.c}
        \lstinputlisting[language=C]{../ncurses-minesweeper/src/main.c}

    \subsection{minesweeper.c}
        \lstinputlisting[language=C]{../ncurses-minesweeper/src/minesweeper.c}

    \subsection{minesweeper.h}
        \lstinputlisting[language=C]{../ncurses-minesweeper/include/minesweeper.h}

    \subsection{gameplay.c}
        \lstinputlisting[language=C]{../ncurses-minesweeper/src/gameplay.c}

    \subsection{gameplay.h}
        \lstinputlisting[language=C]{../ncurses-minesweeper/include/gameplay.h}

    \subsection{navigation.c}
        \lstinputlisting[language=C]{../ncurses-minesweeper/src/navigation.c}

    \subsection{navigation.h}
        \lstinputlisting[language=C]{../ncurses-minesweeper/include/navigation.h}

    \subsection{settings.c}
        \lstinputlisting[language=C]{../ncurses-minesweeper/src/settings.c}

    \subsection{settings.h}
        \lstinputlisting[language=C]{../ncurses-minesweeper/include/settings.h}

    \subsection{outputs.c}
        \lstinputlisting[language=C]{../ncurses-minesweeper/src/outputs.c}

    \subsection{outputs.h}
        \lstinputlisting[language=C]{../ncurses-minesweeper/include/outputs.h}

    \subsection{wins.c}
        \lstinputlisting[language=C]{../ncurses-minesweeper/src/wins.c}

    \subsection{wins.h}
        \lstinputlisting[language=C]{../ncurses-minesweeper/include/wins.h}

    \subsection{audio.c}
        \lstinputlisting[language=C]{../ncurses-minesweeper/src/audio.c}

    \subsection{audio.h}
        \lstinputlisting[language=C]{../ncurses-minesweeper/include/audio.h}

    \subsection{Περιγραφή υλοποιήσης}

    Ο ναρκαλιευτής αυτός χρησιμοποιεί ώς βασική βιβλιοθήκη την ncurses και είναι δομημένος ως εξής:
    Από το main.c καλούνται αρχικά οι συναρτήσεις δημιουργίας των παραθύρων που θα εμφανιστούν
    στην οθόνη και στην συνέχεια καλούνται οι συναρτήσεις δημιουργίας των πινάκων $Μ \times N$,
    για το ναρκοπέδιο και για τον πίνακα που έχει "κρυμμένα" τα κελιά αντίστοιχα.
    Τέλος από την main καλείται η συνάρτηση που θα ξεκινήσει το παιχνίδι και την μουσική.  

    Οι συναρτήσεις για τον ορισμό στηλών, γραμμών, και αριθμό των ναρκών βρίσκονται στο settings.c.
    Τα όρια των διαστάσεω που μπορεί να δώσει ο χρήστης καθορίζονται από το μέγεθος της οθόνης του, το
    οποίο αποθηκεύεται στις μεταβλητές yMax και xMax. Στο wins.c υπάρχουν όλες οι συνάρτησεις δημιουργίας παραθύρων.  

    Στο minesweeper.c εκτελούνται όλες οι συναρτήσεις δημιουργίας πινάκων, τοποθέτησεις ναρκών,
    μέτρημα των ναρκών στα γειτονικά κελιά, καθώς και γέμισμα των κενών θέσεων τους.  

    Έπειτα, στο gameplay.c εκτελείται το παιχνίδι - αρχικά τυπώνεται ο πίνακας και το περίγραμμα που
    υπάρχει ανάμεσα σε κάθε κελί ώστε να είναι πιο εμφανίσιμο και πιο εύχρηστο το παιχνίδι. Προκειμένου
    τα κελιά να τοποθετηθούν στις κατάλληλες θέσεις στον πίνακα, δηλαδή να είναι ανάμεσα στα [ ],
    τα στοιχεία των πινάκων τοποθετούνται κάθε φορά με απόσταση δύο χαρακτήρων στον κάθετο άξονα και 
    τριών χαρακτήρων στον οριζόντιο, το ένα από το άλλο. Με αυτά τα δύο νούμερα προκύπτουν και τέσσερις 
    τύποι, οι οποίοι βοηθάνε στην σωστή προσπέλαση των στοιχείων των πινάκων κατά την διάρκεια του παιχνιδιού,
    και στον υπολογισμό των διαστάσεων του παραθύρου που εμφανίζεται το πεδίο. Οι τύποι είναι οι εξής

    \begin{equation}
        x_{win} = 3cols + 2
    \end{equation}
    \begin{equation}
        x_{board} = \frac{(x_{win}-2)}{3}
    \end{equation}
    \begin{equation}
        y_{win} = rows + 2
    \end{equation}
    \begin{equation}
        y_{board} = y_{win}-2
    \end{equation}

    Οι τύποι (2) και (4) μετατρέπουν την θέση του κέρσορα στην οθόνη σε θέσεις πίνακα.    
    Αφού τυπωθεί στην οθόνη ο πίνακας με κρυμμένα τα στοιχεία του, \textit{ο οποίος είναι στην ουσία ένας 
    $Μ \times N$ πίνακας γεμισμένος με κενά αρχικά}, ξεκινάει το βασικό loop του παιχνιδιού, στο οποίο
    ο χρήστης μετακινείται από κελί σε κελί, επιλέγει την κίνηση που θέλει να κάνει πάνω σε κάθε κελί,
    και είτε χάνει είτε νικάει. Προκειμένου να λειτουργήσει κάτι τέτοιο, μέσα στο loop γίνονται οι εξής
    λειτουργίες: Αρχικά ο κέρσορας μετακίνεται κάθε φορά που και ο χρήστης μετακινείται ώστε να μπορεί να δει 
    σε ποιο κελί βρίσκεται κάθε στιγμή, και στην συνέχεια, μπορούν να εκτελεστούν σε κάθε κελί οι εξής λειτουργίες:
    \begin{itemize}
        \item Άνοιγμα κελιού
        \item Flag ένα κελί
        \item Εξουδετέρωση νάρκης (μόνο αν το κελί είναι ήδη flagged)
    \end{itemize}
    Σε περίπτωση όμως που ο χρήστης κάνει flag ένα κελί που δεν περιέχει νάρκη, και προσπαθήσει να βγάλει την νάρκη,
    ή ανοίξει κελί το οποίο περιέχει νάρκη, θα χάσει. Οπότε το παιχνίδι νικιέται \textit{μόνο} όταν ο χρήστης βγάλει 
    όλες τις νάρκες από το πεδίο, και όχι όταν ανοίξει όλα τα κελιά που δεν έχουνε νάρκες, όπως συμβαίνει στον ναρκαλιευτή των Windows.  

    Συνοπτικά, τα υπόλοιπα αρχεία χειρίζονται ορισμένες λειτουργίες του παιχνιδιού, όπως την κίνηση από κελί σε κελί,
    τον χειρισμό των εμφανίσεων διαφόρων μηνυμάτων στην οθόνη, και τον ήχο. Συγκεκριμένα για τον ήχο χρησιμοποίησα την
    βιβλιοθήκη SDL2 σε συνδυασμό με την βιβλιοθήκη PThread ώστε να μπορούν να εκτελούνται \textit{"ταυτόχρονα"} και οι
    συναρτήσεις χειρισμού ήχου, και το υπόλοιπο παιχνίδι. 

\section{Διευκρινήσεις}

    Αρχικά, λόγω του ότι το πρόγραμμα περιέχει πολλές μεταβλητές και συναρτήσεις θεώρησα καλύτερο να εστιάσω στην λειτουργία 
    του προγράμματος γενικότερα και όχι τόσο στο τι συμβολίζει η κάθε μεταβλητή/συνάρτηση, το οποίο θα μπορούσε
    να γίνει αρκετά κουραστικό. Τα αρχεία και οι συναρτήσεις είναι χωρισμένα έτσι ώστε να εξηγούν από το όνομα τους μόνο ακριβώς ποιες λειτουργίες
    εκτελούνται μέσα σε αυτά ώστε να είναι πιο εύκολη η μετέπειτα κατανόηση, συντήρηση ή βελτίωση του προγράμματος. Bέβαια, κατανοώ ότι
    αυτή η προσέγγιση μπορεί να θεωρηθεί πολύ συνοπτική ή και πρόχειρη.  

    Επίσης, η συγγραφή αυτή τη φορά έγινε στο \LaTeX, λόγω του ότι κάνει αυτόματη στοίχηση, χρωματισμό στους
    κώδικες, και γενικώς πολύ πιο πρακτικό τον χειρισμό μεγάλων εγγράφων, αλλά δεν κατάφερα δυστυχώς να βρω τρόπο
    να αλλάξω την ελληνική γραμματοσειρά σε κάποια πιο εύκολη στο διάβασμα.  
    
    Τέλος, όσο αφορά τον ναρκαλιευτή, θα ήθελα να σημειώσω οτι έχει πολύ περιθώριο για βελτιώση (και λίγο καθαρισμό στον κώδικα),
    και πολλές λειτουργίες, όπως ο ήχος και το multithreading, τις έβαλα περισσότερο πειραματικά, οπότε είναι αρκετά πιθανό να περιέχουν τυχόν
    κακές πρακτικές ή και λάθη, αλλά θεώρησα ότι ήταν μία καλή ευκαιρία για πειραματισμό. Ένα πρόβλημα το οποίο δεν κατάφερα
    να λύσω ήταν το να κάνουν αυτόματο resize τα παράθυρα σε περίπτωση που αλλάξει το μέγεθος του terminal. Μία λειτουργία που 
    θα βάλω στο μέλλον επίσης είναι να δίνεται η επιλογή στον χρήστη να ξαναπαίξει αν θέλει, πιθανώς πατώντας το κουμπί r (restart),
    καθώς και να βάλω χρώματα στους αριθμούς, ώστε να είναι πιο ευδιάκριτοι και να είναι πιο ξεκούραστο στο μάτι το πρόγραμμα.

\section{Εργαλεία}

    \begin{itemize}
        \item Editors: Visual Studio Code, NVim
        \item Compiler: gcc
        \item Βιβλιοθήκες: SDL2, PThread, ncurses
        \item Shell: zsh
        \item OS: Arch Linux
        \item Συγγραφή: \LaTeX
    \end{itemize}

\end{document}

